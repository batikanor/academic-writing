%%%%%%%%%%%%%%%%%%%%%%%%%%%%%%%%%%%%%%%%%
% Masters/Doctoral Thesis 
% LaTeX Template
% Version 1.43 (17/5/14)
%
% This template has been downloaded from:
% http://www.LaTeXTemplates.com
%
% Original authors:
% Steven Gunn 
% http://users.ecs.soton.ac.uk/srg/softwaretools/document/templates/
% and
% Sunil Patel
% http://www.sunilpatel.co.uk/thesis-template/
%
% Editing Authors:
% Gary White
%
% License:
% CC BY-NC-SA 3.0 (http://creativecommons.org/licenses/by-nc-sa/3.0/)
%
% Note:
% Make sure to edit document variables in the Thesis.cls file
%
%%%%%%%%%%%%%%%%%%%%%%%%%%%%%%%%%%%%%%%%%

%----------------------------------------------------------------------------------------
%	PACKAGES AND OTHER DOCUMENT CONFIGURATIONS
%----------------------------------------------------------------------------------------

\documentclass[11pt, oneside]{Thesis} % The default font size and one-sided printing (no margin offsets)
 
\graphicspath{{Figures/}} % Specifies the directory where pictures are stored
\usepackage[square, numbers, comma, sort&compress]{natbib} % Use the natbib reference package - read up on this to edit the reference style; if you want text (e.g. Smith et al., 2012) for the in-text references (instead of numbers), remove 'numbers' 
\hypersetup{colorlinks = true, urlcolor=black, citecolor=gray,	linkcolor=black} % Colors hyperlinks in blue - change to black if annoying # and false
\title{\ttitle} % Defines the thesis title - don't touch this

% Defines the easy list package making bulet points simple
\usepackage[ampersand]{easylist}
\usepackage{tikz}
\usetikzlibrary{arrows,automata}
\usepackage{amssymb}
\usepackage{verbatim}
\usepackage{fancyvrb}
\usepackage{listings}
\usepackage{color}
\usepackage{lipsum}

\definecolor{mygreen}{rgb}{0,0.6,0}
\definecolor{mygray}{rgb}{0.5,0.5,0.5}
\definecolor{mymauve}{rgb}{0.58,0,0.82}

\lstset{ %
	backgroundcolor=\color{white},   % choose the background color; you must add \usepackage{color} or \usepackage{xcolor}
	basicstyle=\footnotesize,        % the size of the fonts that are used for the code
	breakatwhitespace=false,         % sets if automatic breaks should only happen at whitespace
	breaklines=true,                 % sets automatic line breaking
	captionpos=b,                    % sets the caption-position to bottom
	commentstyle=\color{mygreen},    % comment style
	deletekeywords={...},            % if you want to delete keywords from the given language
	escapeinside={\%*}{*)},          % if you want to add LaTeX within your code
	extendedchars=true,              % lets you use non-ASCII characters; for 8-bits encodings only, does not work with UTF-8
	frame=single,	                   % adds a frame around the code
	keepspaces=true,                 % keeps spaces in text, useful for keeping indentation of code (possibly needs columns=flexible)
	keywordstyle=\color{blue},       % keyword style
	language=Octave,                 % the language of the code
	otherkeywords={*,...},           % if you want to add more keywords to the set
	numbers=left,                    % where to put the line-numbers; possible values are (none, left, right)
	numbersep=5pt,                   % how far the line-numbers are from the code
	numberstyle=\tiny\color{mygray}, % the style that is used for the line-numbers
	rulecolor=\color{black},         % if not set, the frame-color may be changed on line-breaks within not-black text (e.g. comments (green here))
	showspaces=false,                % show spaces everywhere adding particular underscores; it overrides 'showstringspaces'
	showstringspaces=false,          % underline spaces within strings only
	showtabs=false,                  % show tabs within strings adding particular underscores
	stepnumber=2,                    % the step between two line-numbers. If it's 1, each line will be numbered
	stringstyle=\color{mymauve},     % string literal style
	tabsize=2,	                   % sets default tabsize to 2 spaces
	title=\lstname                   % show the filename of files included with \lstinputlisting; also try caption instead of title
}
%\usepackage{refcheck}
\ListProperties(Hide=100, Hang=true, Progressive=3ex, Style*=-- ,
Style2*=$\bullet$ ,Style3*=$\circ$ ,Style4*=\tiny$\blacksquare$ )


\usetikzlibrary{shapes,arrows,calc,fit,backgrounds}
\tikzset{sensor/.style={draw, rounded corners, text width=6em, text centered, minimum height=2.5em},
         box/.style = {minimum width=16pt,minimum height=16pt,draw},
         idles/.style = {very thick,draw=violet,text width=6em, align=center, minimum height=12em, rounded corners},
         line/.style = {,>=latex,->,draw=blue!50}}
\def\blockdist{4}
\def\tdist{2.25}

\DefineVerbatimEnvironment{VerbatimTest}{Verbatim}%
{showspaces,showtabs,commandchars=\\\{\}}% showspaces and showtabs only for visualizing


\begin{document}
%\nocite{*}
\frontmatter % Use roman page numbering style (i, ii, iii, iv...) for the pre-content pages

\setstretch{1.3} % Line spacing of 1.3

% Define the page headers using the FancyHdr package and set up for one-sided printing
\fancyhead{} % Clears all page headers and footers
\rhead{\thepage} % Sets the right side header to show the page number
\lhead{} % Clears the left side page header

\pagestyle{fancy} % Finally, use the "fancy" page style to implement the FancyHdr headers

\newcommand{\HRule}{\rule{\linewidth}{0.5mm}} % New command to make the lines in the title page

% PDF meta-data
\hypersetup{pdftitle={\ttitle}}
\hypersetup{pdfsubject=\subjectname}
\hypersetup{pdfauthor=\authornames}
\hypersetup{pdfkeywords=\keywordnames}

%----------------------------------------------------------------------------------------
%	TITLE PAGE
%----------------------------------------------------------------------------------------

\begin{titlepage}
\begin{center}

\textsc{\LARGE \univname}\\[1.5cm] % University name\\
\includegraphics[height = 5.5cm]{Pictures/trinity-logo}\\[0.5cm]

\HRule \\[0.4cm] % Horizontal line
%{\huge \bfseries \ttitle}\\[0.4cm] % Thesis title
{\huge \bfseries Thesis Title 3 }\\[0.4cm] % Thesis title
\HRule \\[1.5cm] % Horizontal line

\begin{minipage}{0.4\textwidth}
\begin{flushleft} \large
\emph{Author:}\\ \authornames % Author name - remove the \href bracket to remove the link
\end{flushleft}
\end{minipage}
\begin{minipage}{0.4\textwidth}
\begin{flushright} \large
\emph{Supervisor:} \\
\href{http://people.tcd.ie/Profile?Username=moregan}{\supname} % Supervisor name - remove the \href bracket to remove the link  
\end{flushright}
\end{minipage}\\[3cm]
 
\large \textit{A report submitted in fulfilment of the requirements\\ for the degree of \degreename}\\[0.3cm] % University requirement text
\textit{in the}\\[0.4cm]
%\groupname
\deptname\\[2cm] % Research group name and department name
 
{\large \today}\\[4cm] % Date
%\includegraphics{Logo} % University/department logo - uncomment to place it
 
\vfill
\end{center}

\end{titlepage}

%----------------------------------------------------------------------------------------
%	SUMMARY
%----------------------------------------------------------------------------------------


%----------------------------------------------------------------------------------------
%	LIST OF CONTENTS/FIGURES/TABLES PAGES
%----------------------------------------------------------------------------------------

\pagestyle{fancy} % The page style headers have been "empty" all this time, now use the "fancy" headers as defined before to bring them back

\lhead{\emph{Contents}} % Set the left side page header to "Contents"
\tableofcontents % Write out the Table of Contents

\lhead{\emph{List of Figures}} % Set the left side page header to "List of Figures"
\listoffigures % Write out the List of Figures

\lhead{\emph{List of Tables}} % Set the left side page header to "List of Tables"
\listoftables % Write out the List of Tables


%----------------------------------------------------------------------------------------
%	THESIS CONTENT - CHAPTERS
%----------------------------------------------------------------------------------------

\mainmatter % Begin numeric (1,2,3...) page numbering

\pagestyle{fancy} % Return the page headers back to the "fancy" style

% Include the chapters of the thesis as separate files from the Chapters folder
% Uncomment the lines as you write the chapters

% Chapter 1

\chapter{Univariate Analysis} % Main chapter title

\label{Chapter1} % For referencing the chapter elsewhere, use \ref{Chapter1} 

\lhead{1 - \emph{Analysis}} % This is for the header on each page - perhaps a shortened title

%----------------------------------------------------------------------------------------
\section{Introduction}
\cite{10.1109/64.21887} and \cite{frankel2014effects} show that x, y, x.
\lipsum

% Chapter 2

\chapter{Interesting Topic} % Main chapter title


\label{Chapter2} % For referencing the chapter elsewhere, use \ref{Chapter1} 

\lhead{2 - \emph{Interesting Topic}} % This is for the header on each page - perhaps a shortened title

\lipsum 
% Chapter 3

\chapter{Ensemble Methods} % Main chapter title

\label{Chapter3} % Change X to a consecutive number; for referencing this chapter elsewhere, use \ref{ChapterX}

\lhead{3 - \emph{Ensemble Methods}} % Change X to a consecutive number; this is for the header on each page - perhaps a shortened title
\lipsum
% Chapter 4

\chapter{Comparison between models} % Main chapter title

\label{Chapter4} % Change X to a consecutive number; for referencing this chapter elsewhere, use \ref{ChapterX}

\lhead{4 - \emph{Comparison between models}} % Change X to a consecutive number; this is for the header on each page - perhaps a shortened title

\lipsum 
% Chapter 5

\chapter{Imaginative Contribution} % Main chapter title

\label{Chapter5} % Change X to a consecutive number; for referencing this chapter elsewhere, use \ref{ChapterX}

\lhead{5 - \emph{Imaginative Contribution}} 
\lipsum 
%\input{Chapters/Chapter6} 
%\input{Chapters/Chapter7} 

%----------------------------------------------------------------------------------------
%	THESIS CONTENT - APPENDICES
%----------------------------------------------------------------------------------------

\addtocontents{toc}{\vspace{2em}} % Add a gap in the Contents, for aesthetics

\appendix % Cue to tell LaTeX that the following 'chapters' are Appendices

% Include the appendices of the thesis as separate files from the Appendices folder
% Uncomment the lines as you write the Appendices

% Appendix A

\chapter{Included Code} % Main appendix title

\label{AppendixA} % For referencing this appendix elsewhere, use \ref{AppendixA}

%\lhead{Appendix A - \emph{Inference Statistics}} % This is for the header on each page - perhaps a shortened title
The full source code is available on the CD, the code is broken into a number of different folders which deal with the weather analysis, sentiment analysis, trend following program in Matlab and finally the visualisations that were produced by using the D3 library. I have made an effort to make the specific parts of the project as accessible as possible by including a detailed ReadMe markdown file in the top level directory. I have also tried to comment and layout the code and data in the most intuitive way possible. 

%\input{Appendices/AppendixB}
%\input{Appendices/AppendixC}

\addtocontents{toc}{\vspace{2em}} % Add a gap in the Contents, for aesthetics

\backmatter

\label{Bibliography}

\lhead{\emph{Bibliography}} % Change the page header to say "Bibliography"

\bibliographystyle{unsrtnat} % Use the "unsrtnat" BibTeX style for formatting the Bibliography

\bibliography{Bibliography} % The references (bibliography) information are stored in the file named "Bibliography.bib"






\end{document}  