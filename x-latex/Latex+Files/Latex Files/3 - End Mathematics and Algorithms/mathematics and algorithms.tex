\documentclass[a4paper, twocolumn]{article}
\usepackage{lipsum}
\usepackage{amsmath}
\usepackage[algo2e]{algorithm2e}
\usepackage{algorithm}
\usepackage{algorithmic}
\usepackage{listings}

\title{Mathematics and Algorithms}
\author{Gary}

\begin{document}
\maketitle	

\begin{abstract}
	\lipsum[1]
\end{abstract}

\tableofcontents

\section{Mathematics}
This is an equations that is in a line $x_1 = y^2$

In Equation \ref{eq:binom} we see the binomial formula.  

\begin{equation}
\label{eq:binom}
\sum^{n}_{i=0}
\binom{n}{i}a^{i}b^{n-i}
= (a+b)^{n}
\end{equation}

\subsection{Fractions}

\begin{equation}
\frac{a + b}{c}
\end{equation}

\begin{equation}
\sqrt{2} - 1
= \cfrac{1}{2 +
	\cfrac{1}{2 +
		\dotsb}}\,.
\end{equation}

\subsection{Align Environment}

\begin{align}
z &= a + b + c \\
&= \sum_{min}^{max} x^2 \\
\nonumber
    &= f + r
\end{align}

\subsection{Integration and Differentiation}
\[ \int^{b}_{a}
3 x^{2}\,d x
= \left. x^{3}
\right\rvert^{b}_{a}
= b^{3} - a^{3}\,. \]


Let $z = x^{2} + xy$, then
\[ \frac{\partial z}
{\partial x}
= 2x + y\,. \]

\subsection{Braces}
\[ x^{k} =
\underbrace
{1 \times x
	\times x \times
	\dotsb \times x}
_{\text{$k$˜times
		$\times x$}} \,. \]

\subsection{Conditionals}
\[ n! = 
\begin{cases} 1  & \text{if $n = 0$}\,; \\
(n-1) ! \times n & \text{if $n > 0$}\,.
\end{cases} \]

\section{Algorithms}

\subsection{algorithm2e}
\lipsum[1]

\begin{algorithm2e}[tbp]
	\Switch{order}{
		\uCase{bloody mary}{
			Add tomato juice\;
			Add vodka\;
			\If{vodka $\le$ 2l}{buy more} 
			break\;
		}
		\uCase{hot whiskey}{
			Add whiskey\;
			Add hot water\;
			Add lemon and cloves\;
			Add sugar or honey to taste\;
			break\;
		}
		\Other{Serve water\;}
	}
\end{algorithm2e}

\lipsum[1-2]

\subsection{algorithmic}
\begin{algorithm}
	\begin{algorithmic}[1]
		\REQUIRE{Input}
		\ENSURE{Output} 
		\IF{some condition is true}\label{line:if}
		\STATE do some processing
		\ELSIF{some other condition is true}
		\STATE do some different processing
		\ELSE 
		\STATE do the default actions \label{line:default}
		\ENDIF
	\end{algorithmic}
	\caption{My Algorithm}
	\label{algo:mine}
\end{algorithm}

As seen in Algorithm \ref{algo:mine}. In Line \ref{line:if} we can see the if statement. In Line \ref{line:default} we present the default statements. 


\subsection{listings}
\lstset{language=c++}
\lstset{caption=Some C++ Code}
\begin{lstlisting}[frame=single]{}
for(i = 0; i < 10; i++){
// increment the pointer
*p++ = i;
}
\end{lstlisting}


\end{document}
















